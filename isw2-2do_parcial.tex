\documentclass[a4paper,spanish]{article}

\usepackage[spanish,activeacute]{babel}
\usepackage{moreverb}
\usepackage{fancyhdr}
\usepackage{graphicx}
\usepackage{multicol}

\oddsidemargin 0in
\textwidth 6.2in
\topmargin 0in
\addtolength{\topmargin}{-.5in}
\textheight 10in
\parskip=1ex
\pagestyle{fancy}
%usar el segundo nivel de enumeracion con letras
%\renewcommand{\labelenumii}{\alph{enumii}. }

\newcommand{\rev}[0]{\marginpar{REVISAR}}
\newcommand{\nohecho}[0]{\marginpar{NO HECHO}}

\newcommand{\tab}[0]{\hspace*{0.5cm}}

\newcommand{\then}[0]{\ensuremath{\Rightarrow}~}

\newenvironment{items}{
		\vspace*{-\topsep}
		\begin{itemize} 
		\addtolength{\itemsep}{-0.5\baselineskip}
		}{\end{itemize}\vspace*{-\topsep}}

\newenvironment{numbers}{
		\vspace*{-\topsep}
		\begin{enumerate} 
		\addtolength{\itemsep}{-0.5\baselineskip}
		}{\end{enumerate}\vspace*{-\topsep}}

%noindent en todos lados
\parindent=0in 

\lhead{Ingenier\'ia de Software II}
\rhead{Apunte de repaso segundo parcial}

\cfoot{$\thepage$ de \pageref{theend}}


\begin{document}

Disclaimer: Este apunte no es autocontenido y fue pensado como un repaso de los conceptos, 
no para aprenderlos de aqu'i directamente.
%\begin{multicols}{2}
%\tableofcontents
%\end{multicols}

\section{Metodolog\'ias \'Agiles}

Las metodolog'ias 'agiles cambian el foco respecto de las tradicionales:\\
\begin{tabular}{lcl}
Individuos y sus interacciones &$>$& herramientas + procesos \\
Construir software &$>$& documentaci'on completa \\
Colaboraci'on con el cliente &$>$& negociaci'on de contratos \\
Responder ante el cambio &$>$& el seguimiento del plan
\end{tabular}

Principios b'asicos:
\begin{items}
\item Visibilidad
\item Inspecci'on (detectar desvios a tiempo)
\item Adaptaci'on
\end{items}

Otras caracter'isticas: Emp'iricas, ``livianas'', adaptativas, r'apidas pero
	no apuradas, emergentes (!), exponen p'erdidas, centradas en el cliente,
	decisiones en niveles bajos, self-organized, time boxed

Se asume que va a haber cambios, se toma el mismo como forma de acercarse al
deseo del cliente. Igualmente tiene mecanismos de control y estructura.

No sirven en equipos grandes o distribuidos (baja comunicaci'on), si hay poca
	experiencia, en donde todo ya anda bien y cuando es dificil definir tarea
	terminada.

\subsection{SCRUM}

Pasos de SCRUM: 
\begin{numbers}
\item Product backlog: Conjunto de items que son user stories 
	(reqs func y no func). Es lo que se sabe que se quiere que este en
	el producto final.
\item Sprint planning: Seleccionar subconjunto del product backlog seg'un
	priorizaci'on del cliente y dividir	los items en tareas seg'un cree
	el team.
\item Sprint backlog: Items + taskboard
\item SPRINT (daily standups): Personas se autoasignan tareas del taskboard y
	las hacen. Daily standups: Cerditos dicen lo que hicieron, los pollitos 
	miran. Sirve para dar comunicaci'on y que todos vean la big picture (las
	cosas se retrasan un dia por vez por falta de comunicaci'on). El taskboard
	se actualiza todos los dias con estimaci'on de tiempo de tareas (1-16 hs).
	Tareas largas hacen daily standups improductivos (todos dicen ``lo mismo
	que ayer'').
\item Producto incrementado 
\item Sprint review: Con el cliente, managment y product owner. Feedback.
\item Sprint retrospective: Sin el cliente, mejorar el proceso.
\end{numbers}

Roles de SCRUM:
\begin{items}
\item Scrum master: Usa la lista de impediments para resolverlos. Asegura el
	cumplimiento del proceso.
\item Product owner: ``Abogado'' del cliente. Su objetivo es maximizar ROI.
	Prioriza los reqs, resume la input del cliente. Puede cambiar reqs entre
	sprints, puede aceptar/rechazar.
\item Team: Los que hacen. No hay roles, solo skills (desarrollo, testing,
	analista funcional). Estima y define sprint.
\end{items}

Inicio: Product backlog con items iniciales, funcionalidad y \#sprints hasta
	el release, evaluaci'on de riesgos, capacitiaci'on de la gente (incluido
	cliente) en SCRUM. \\
Fin: Cierre release, instalaci'on, documentaci'on y entrenamiento.

Especificaci'on: Epics (grupos de funcionalidad), user stories (detalle de 
	epics) que incluyen CU y casos de aceptaci'on.

\subsection{eXtreme Programming}

Roles vs responsabilidades: Cualquiera podr'ia tomar cualquier rol: 
	Desarrollador, cliente, tester, tracker, consultor, coach, manager.
	
Valores principales: Comunidad, simplicidad, feedback, coraje y respeto.

Pr'acticas:
\begin{items}
\item Planning process
\item Releases chicos, iteraciones cortas
\item Met'aforas (lenguaje com'un con el cliente)
\item Dise~no simple.
\item Testing continuo como parte integral del desarrollo
\item Refactoring
\item TDD (test-driven development): Escribir test antes de desarrollar, 
	correrlo para que falle, arreglarlo.
\item Pair programming \then Collective ownership (rotar pares y roles dentro
	del par)
\item Collective ownership
\item Integraci'on continua
\item Semana de 40 horas
\item Cliente on-site
\item Est'andar de c'odigo
\end{items}

Ciclo dorado: \\
requerimientos cambiantes \then dise~no emergente \then refactoring \then
testing continuo \then pair programming \then collective ownership \then
on-site client

\begin{tabular}{l|l|l|l}
& Exploration & Commitment & Steering \\
\hline
Release planning & stories/priorizar & alcance release & ajustes +/- reqs \\
\hline
Iteration planning & task cards & asignar tasks, estimaci'on & iteraci'on de
	trabajo y contraste con stories
\end{tabular}

\section{Software Configuration Managment (SCM)}

Control de cambios: Custodiar integridad y acompa~nar los cambios con control.

Incremento en complejidad: \\
\tab Soft: Multiplataforma, $>$ tama~no, interfases, OTS \\
\tab Env: Tama~no, distribuci'on geogr'afica del team, frecuentes releases,
	cambios SO/hard.
	
\begin{numbers}
\item Id y almacenar artefactos en repositorio seguro
\item Controlar y auditar cambios en artefactos (aumenta el control al 
	avanzar el proy)
\item Organizar artefactos en SCM components:
	\begin{items}
	\item Contribuye a preservar integridad
	\item Reduce complejidad (D\&C)
	\item Facilita determinaci'on del nivel de calidad
	\item Aumenta el c'odigo compartido y el reuso
	\end{items}
\item Crear baseline para cada milestone del proyecto
	\begin{items}
	\item + Baselines al final de iteraci'on o etapa de entrega
	\item + Reproducibilidad (versiones). Id problemas introducidos en 
		releases nuevos.
	\item + Traceability (unif. artefactos)
	\item Reporting / logging (facilita debug y notas de release, diffs)
	\end{items}
\item Registro y seguimiento de pedidos de cambio (change request managment)
\item Mantener entorno de trabajo estable
\item Soportar cambios concurrentes sobre artefactos
\item Integraci'on en forma temprana y frecuente
\item Asegurar que es posible reproducir releases (mantenimiento)
\end{numbers}

Integraci'on: Merge o assembly.

Consistencia: Turn-taking, split-merge o copy-merge.

\section{Calidad}

Definiciones:
\begin{items}
\item Aptitud para uso
\item Ausencia de defectos
\item Satisfacci'on de requerimientos
\item Nivel hasta el que un producto tiene un conjunto de atributos
\end{items}

V\&V:\\
\tab Validaci'on: Producto correcto (requerimientos correctos)
\tab Verificaci'on: C'odigo correcto respecto de los requerimientos

Error (humano) \then Defecto (en el c'odigo) \then Falla (observable)

Modelo en V:\\
\begin{tabular}{ll}
Requerimientos de sistema    & \tab\tab\tab\tab Integraci'on del sistema \\
\tab Requerimientos de soft  & \tab\tab\tab Test de aceptaci'on \\
\tab\tab Dise~no arquitectura             & \tab\tab Integraci'on de software \\
\tab\tab\tab Dise~no detallado        & \tab Test de componentes \\
\tab\tab\tab\tab Implementaci'on en c'odigo     & Test de unidad \\
\end{tabular}

Testing: El crear tests ya ayuda por el proceso mental de pensarlos. \\
>Cu'ando termina? Heur'isticas usuales: No hay fallas, porcentaje aceptable de
fallas, \# defectos hallados $\sim$ \# defectos estimados, error seeding.

Tipos de test:
\begin{items}
\item De unidad
\item De integraci'on
\item De sistema
\item De aceptaci'on: Baasado en requerimientos, dise~nado y ejecutad por
	usuarios
\item De vol'umen y performance (borde)
\item De stress (exceso)
\item De regresi'on (luego de un cambio, ver que no se introdujeron fallas)
\item $\alpha$ (usuario interno) $\beta$ (usuario real)
\item De usabilidad
\item Funcional: Tipos de fallas, sincronizados con la especificaci'on que les
	di'o origen.
\end{items}

Revisiones
\begin{items}
\item Walkthrough: Presentador (conoce a fondo el producto) + asistentes
\item Revisiones (formalidad intermedia)
\item Inspecciones de c'odigo (mayor formalidad): Lector, revisor, autor, 
	registrador y moderador. Obtiene como resultado de cada error: ubicaci'on,
	descripci'on, severidad y tipo (interfaz, datos, l'ogica). Objetivos
	secundarios:
	\begin{items}
	\item Aumentar visibilidad
	\item Buscar consenso
	\item Favorecer trabajo en equipo
	\item Obtener datos para las m'etricas
	\item Lector, revisor, 
	\end{items}
	Hay que hacerla cuando el testing es intermedio, al principio hay 
	demasiados bugs y luego de testing exhaustivo no hay lugar para el 
	trabajo.
\end{items}

\section{M\'etricas}

Objetivo previamente esclarecido: \\
Identificar objetos del dominio \then definir escala (relacionada con los 
	objetos)

>Para qu'e usar m'etricas?
\begin{items}
\item Extender y modelar procesos y productos
	\begin{items}
	\item Comparar lineas de base ($\neq$ versiones)
	\item Predecir el efecto de un cambio
	\item Calcular esfuerzo/recursos/tipos de error comunes
	\end{items}
\item Administraci'on: Estimaci'on, detectar desvios, averiguar relaci'on 
	entre par'ametros (ej: erroes/linea de c'odigo)
\end{items}

Las m'etricas pueden ser de producto, de proyecto o de proceso.

\subsection{Goal Question Metric (GQM)}

Goal: Parte conceptual, objetivo. \\
Question: Parte operacional, modelo. \\
Metric: Parte cuantitativa, m'etrica.

Construir software \then proceso de ingenier'ia \then retroalimentaci'on y
	evaluaci'on \then objetivos medibles dirigidos por un modelo.
	
El objetivo depende de la perspectiva (cliente $\neq$ proyeto $\neq$ 
	corporaci'on).
	
\begin{numbers}
\item Definir objetivo de proyecto y corporativo: Operacionalmente refinado
	con preguntas medibles.
\item >Qu'e datos ``hablan'' sobre ese objetivo?
\item Definir m'etricas de respuesta y universo de interpretaci'on
\item Medir
\item Interpretar
\end{numbers}

Fallas: Medir demasiado (medir indicadores irrelevantes), usar resultados para
	premio o castigo, cortoplazismo, esfuerzo de medici'on no homog'eneo,
	no usar el resultado de las mediciones.
	
\subsection{Alta madurez}

Alta madurez: Gesti'on cuantitativa de procesos y productos. \\
\tab Seg'un CMMI: Uso sistem'atico de mediciones de los procesos para predecir
	costos, cronograma y defectos y establecer un rango de performance.
	
\begin{tabular}{rcccl}
& Mejorar & \\
$\swarrow$ & & $\nwarrow$ \\
Definir & Controlar & $\longleftarrow$ Medir \\
$\searrow$ & $\downarrow$ & $\nearrow$ \\
& Ejecutar &
\end{tabular}

Definiciones de procesos:
\begin{items}
\item Performance (del proceso): Calidad, costo, tiempo
\item Capacidad: Rangos de atributos
\item Control cuantitativo (no estad'istico)
\item Control estad'istico
\item Estabilidad: Predicabilidad
\end{items}

CMMI N4:
\begin{items}
\item Baseline de performance y modelos
\item Manejar cuantitativamente
\item Manejar performance de subprocesos estad'isticamente
\end{items}

Saber como se comporta el proceso con n'umeros: \% de tiempo de gesti'on. \# de
	bugs en un peer review. \then Planificar (qu'e, c'omo y cu'ando medir)
	\then Acumular n'umeros de muchos proyectos \then Predecir 
	estad'isticamente.

Trabajo: Serie de procesos interconectados, todos pueden fallar.
\begin{items}
\item Entender pasado cuantitativamente
\item Controlar presente cuantitativamente
\item Predecir futuro cuantitativamente
\end{items}

Causas: comunes (esperables) vs especiales (no esperables) de variaci'on.

Mediciones a varios niveles: Empresa, l'inea de producto, proyecto, proceso,
	subproceso. Mas alto, menos preciso el an'alisis.
	
Control estad'istico: Problemas: Muy dificil distinguir causas comunes de 
	especiales. Si o si tienen que ser proyectos homog'eneos (tecnolog'ia,
	tama~no).
	
SPC
\begin{items}
\item Elegir procesos adecuadamente, no combinar datos de organizaciones $\neq$
\item Ajustar l'imites
\item Efecto hawthorne (el medir perturba, incertidumbre)
\item Comportamiento disfuncional (se trabaja para la m'etrica)
\end{items}

CMMI N5: Pasar de detectar a prevenir defectos. Mejorar segun entendimiento
	cuantitativo.
\begin{items}
\item Detectar causas de defectos \then atacarlas
\item Seleccionar mejores \then implementarlas (incluye medir sus efectos)
\end{items}

Al analizar causas de defectos: Identificar, clasificar y priorizar. 
	Identificar efecto y sus posibles soluciones. Prevenir nueva aparici'on
	(hacer pilotos con objetivos de performance y medirlos, si andan, 
	extender).

Alta madurez en CMMI: N4: Comprensi'on y administraci'on de variaci'on de 
	procesos para obtener calidad. N5: Mejorar tecnolog'ias innovadoras. 
	Implementarlas utilizando conocimiento cuantitativo.

\section{Architecture Trade-off Analysis Method (ATAM)}

Utilidad: Valoraci'on del stakeholder (SH) \\
Valor: Caracterizaci'on del impacto de las decisiones

Beneficios:
\begin{items}
\item Fuerza la preparaci'on de material para review
\item Captura motivaciones de la arquitectura
\item Determinaci'on temprana de problemas
\item Valida requerimientos
\item Mejora arquitectura
\end{items}

Dificultad: Muchos SH \then diferencias valoraciones de utilidad \then 
	v'inculo decisiones de dise~no - visibilidad SH
	
Precondiciones:
\begin{items}
\item Requerimientos y objetivos de arquitectura articulados
\item Alcance definido
\item Cost-efectiveness
\item Miembros clave disponibles (SH de la arquitectura y project-decision 
	makers)
\item Team de evaluaci'on competente (3 a 5 externos, parte de QA o ad-hoc)
\item Expectativas gestionadas
\end{items}

Fases: \\
0. Contratos y NDAs, informaci'on inicial requerida (introducci'on) \\
1. Evaluaci'on. Team + decision makers (uniformizar) \\
2. Evaluaci'on. Team + decision makers + SH (consenso) \\
3. Segumiento, emisi'on de resultados, analisis post-mortem

'Arbol de utilidad: Tiene objetivos de calidad en los nodos y escenarios en 
	las hojas.

\begin{numbers}
\item Presentar ATAM y 'arboles de utilidad
\item Presentar drivers de negocio: Requerimientos funcionales y de calidad
\item Presentar arquitectura: Restricciones t'ecnicas, otros sistemas para
	interactuar
\item Identificar enfoques de la arquitectura: Aspectos clave QA
\item Generar 'arbol de utilidad de atributos de calidad.
\item Analizar enfoques de la arquitectura: Identificar enfoques QA 
	prioritarios, generar preguntas para QA prioritarios, identificar riesgos,
	no-riesgos, puntos sensibles y compromisos.
\item Brainstorming: Priorizar escenarios. Agregar escenarios al 'arbol 
	inicial (los iniciales sirven de ejemplo)
\item Analizar enfoques de arquitectura sobre lo surgido en el punto anterior
\item Presentar resultados. Salidas:
	\begin{items}
	\item Presentaci'on concisa de arquitectura
	\item Articulaci'on de los objetivos del negocio
	\item Requerimientos de calidad expresados en escenarios
	\item Mapeo de requerimientos de calidad con decisiones arquitect'onicas
	\item Identificaci'on de puntos sensibles, trade-offs, risks y non-risks
		(asunciones)
	\item Risk themes
	\end{items}
\end{numbers}

\begin{tabular}{lccccccl}
$\longrightarrow$ & Drivers & $\rightarrow$ & QA & $\rightarrow$ & Escenarios & $\longrightarrow$ & Analisis \\
$\longrightarrow$ & Arquitectura & $\rightarrow$ & Approach & $\rightarrow$ & Decisiones & $\longrightarrow$ & $\downarrow$ \\
$\uparrow$ & & & & & & & $\downarrow$ \\
$\uparrow$ & & & & & Trade-offs & $\longleftarrow$ & $\downarrow$ \\
$\uparrow$ & & & & & Puntos sensibles & $\longleftarrow$ & $\downarrow$ \\
$\uparrow$ & & & & & No-riesgos & $\longleftarrow$ & $\downarrow$ \\
$\nwarrow$ & $\longleftarrow$ & $\leftarrow$ & $\longleftarrow$ & $\leftarrow$ & Riesgos & $\longleftarrow$ & $\downarrow$ \\
\end{tabular}

Conclusiones:
\begin{items}
\item $\uparrow$ Perspectiva de calidad para usuario
\item $\uparrow$ Priorizaci'on de dificultad y utilidad
\item $\uparrow$ Consenso sobre riesgos y no-riesgos
\item $\uparrow$ Base documentada del modelo
\item $\downarrow$ No evaluo costo
\item $\downarrow$ No variaciones de escenarios (discreto)
\item $\downarrow$ No es cuantitativo
\end{items}

Variaciones de escenarios: Comparar distintos posibles valores de respuesta
	(no binario). Curva utilidad/respuesta. Expresar concretamente QA.

\section{Administraci'on de requerimientos y SQA}

Areas de conocimiento: \\
\begin{items}
\item Ingenier'ia de requerimientos
	\begin{items}
	\item Development
		\begin{items}
		\item Elicitation
		\item An'alisis
		\item Especificaci'on
		\item Validaci'on
		\end{items}
	\item Managment
	\end{items}
\end{items}

Deseable de los reqs: Completo, correcto, factible, verificable, necesario, 
	priorizado, no ambig"uo.

Managment: Procesos relacionados con requerimientos: Cambios, agregados, 
	asignaci'on a iteraciones. \\
Managment de cambios sobre artefactos $\in$ linea de base: Pasos de un SH 
	desde un change request a un artefacto (en particular requirement change)
	hasta una decisi'on.
	
Tipos de cambio a requerimientos:
\begin{items}
\item Cambio: En contra o mayor a requerimiento aprobado
\item Mejoras: ``Peque~nas'' no expresadas en los requerimientos (= implican
	esfuerzo)
\end{items}

Gesti'on de cambio: Es importante el workflow. Objetivo: Controlar cambios
	(no evitarlos!). Las metodolog'ias 'agiles tambi'en tienen!

CCB (configuration control board). Equipo que toma decisiones de cambios. 
	Puede haber varios seg'un el tipo de cambio e incluso una jerarqu'ia.
	
Desarrollo iterativo: El cambio es bienvenido! Al prinicipio hay muchos CU
	nuevos pero al final hay mayormente cambios a los existentes. \then
	>Como especifico cambios? >Seguimiento? >Alcance? $\Delta$CU: Especifico
	solo la diferencia.
	
SQA: Conjunto de tareas que:
\begin{items}
\item Eval'uan objetivamente la ejecuci'on de procesos y los entregables
\item Identifican, documentan y aseguran el correcto trato de desvios
\item Provee feedback sobre QA
\end{items}

Objetivos de QA: (no garantiza que todo ande bien!)
\begin{items}
\item Dar visibilidad a la gerencia (especialmente desvios)
\item Asegurar cumplimiento del proceso
\item Ayudar a ``poner la calidad'' (a trav'es de revisiones)
\end{items}

\label{theend}
\end{document}
